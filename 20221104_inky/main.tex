\documentclass{article}
\usepackage[utf8]{inputenc}
\usepackage[T1]{fontenc}
\usepackage[basque]{babel}
\usepackage{hyperref}
\usepackage{graphicx}
\usepackage{longtable}
\usepackage{booktabs}
\usepackage[table]{xcolor}

\title{Inky}
\author{}
\date{}

\begin{document}

\maketitle

%\tableofcontents

\begin{center}
\begin{longtable}{|p{6cm}|p{6cm}|}
  \toprule
  \cellcolor{lightgray}{\textbf{Título}} &
  \cellcolor{lightgray}{\textbf{Izenburua}}\\
  
  \midrule
  Inky, el pulpo que encontró su camino de vuelta al océano&
  Inky, olagarroa bera, ozeanora itzultzeko bere bidea aurkitu zuena\\

  \midrule
  \cellcolor{lightgray}{\textbf{Párrafo}} &
  \cellcolor{lightgray}{\textbf{Paragrafoa}}\\
  
  \midrule
  Inky ha protagonizado cientos de titulares en la prensa internacional por una hazaña que llevó a cabo hace unos meses.&
  Inky nazioarteko prentsan ehundaka lerroburuetako protagonista izan da balentria bategatik zeina orain dela hilabete batzuk aurrera eraman baitzuen.\\

  \midrule
  Por razones que se desconocen, los responsables del Acuario Nacional de Nueva Zelanda, en Napier, han decidido que era el momento de explicar que la desaparición de este pulpo de su exhibición se debió a una fuga voluntaria e inteligentemente planificada.&
  Arrazoi batzuengatik zeintzuak ez baitira ezagutzen, Zelanda Berriko Akuario Naziotiarreko arduradunek, Napier hirian, erabaki dute ezen unea zela azaltzeko ezen olagarro hau bere erakus-akuariotik desagertzearen zioa ihesaldi bat izan zela, berau borondatezkoa eta adimentsuki plangintzatua izan zena.\\

  \midrule
  Ellos, claro está, no lo han explicado así...&
  Eurek, jakina denez, ez dute horrela azaldu...\\

  \midrule
  Su historia, como la de tantas otras que día tras día luchan contra su encierro en zoos y acuarios, es contada como una anécdota, una casualidad, en este caso fruto de la instintiva curiosidad y asombrosa flexibilidad de su especie.&
  Bere historia, nola eguna joan eguna etorri zoo eta akuarioetan euren itxiturapeko egoeraren aurka borrokatzen duten beste hainbatena bezala, anekdota bat bailitzan kontatua da, kasualitate bat, kasu honetan bere espeziearen instintiar jakinmin eta malgutasun txundigarriaren fruitua bailiran.\\

  \midrule
  Lo comparan con la fuga ficticia de la película Buscando a Nemo, canto a la libertad del cual muchos no aprendieron nada.&
  Nemoren Bila filmeko fiktiziozko ihesaldiarekin alderatzen dute, askatasunari kantua dena, zeinetik askok ezer ez baitzuten ikasi.\\

  \midrule
  Se permiten el lujo de bromear sobre la rebeldía del pulpo comentando que ``ni siquiera dejó una nota''.&
  Baimentzen diote euren buruari olagarroaren errebeldiaren gainean txantxak egiteko luxua komentatuz ezen ``ez zuen ezta sikiera ohar bat utzi''.\\

  \midrule
  Esperamos que Inky esté riendo más fuerte en lo profundo del océano, sabiendo que ha burlado a sus captores y que su vida vuelve a ser sólo suya.&
  Espero dugu ezen Inky ozeanoaren sakonean are eta ozenago barre egiten ariko dela, jakinez ezen bere bahitzaileak burlatu dituela eta berriz ere bere bizitza berea bakarrik izatera itzuli dela.\\

  \midrule
  \cellcolor{lightgray}{\textbf{Párrafo}} &
  \cellcolor{lightgray}{\textbf{Paragrafoa}}\\
  
  \midrule
  Porque, si es cierto lo que cuentan, echamos en falta algo en el cómo lo cuentan.&
  Zergatik, kontatzen dutena zuzena bada, zerbaiten falta sumatzen dugu zer hau nola kontatzen duten.\\

  \midrule
  Algo que para nosotras es lo más importante: Inky había conocido la libertad, se la habían robado, y utilizó todos los recursos a su alcance para recuperarla.&
  Zerbait zeina, guretzat, gauza garrantzitsuena baita: Inky-k askatasuna ezagutu zuen, lapurtu egin zioten, eta bere eskura zituen baliabide guztiak erabili zituen askatasuna berreskuratzeko.\\

  \midrule
  \cellcolor{lightgray}{\textbf{Párrafo}} &
  \cellcolor{lightgray}{\textbf{Paragrafoa}}\\
  
  \midrule
  Inky vivía libre en el Pacífico.&
  Inky Ozeano Barean aske bizi zen.\\

  \midrule
  Hace unos años, quedó atrapado en una trampa para cangrejos instalada en un arrecife.&
  Duela urte batzuk, uharri batean instalatutako karramarroentzako tranpa batean harrapatuta geratu zen.\\

  \midrule
  El pescador que lo encontró, en lugar de devolverle al océano, que habría sido lo más justo y sencillo, decidió entregarlo al acuario.&
  Aurkitu zuen arrantzaleak, ozeanora itzuli beharrean, zeina gauzarik zuzen eta errazena izango baitzen, akuarioari ematea erabaki zuen.\\

  \midrule
  O así es como dicen que se hicieron con él y lo encerraron en un tanque para exhibirlo a los visitantes y ganar dinero a su costa.&
  Edo horrela esaten dute ezen berataz jabetu zirela eta tanke batean itxipetu zutela bisitariei erakusteko eta bere kontura dirua irabazteko.\\

  \midrule
  La excusa que dieron para mantenerlo cautivo era que estaba en mal estado, ya que su vida en el arrecife, luchando día a día para obtener alimento, era mucho más dura que la que tenía en su jaula de cristal.&
  Bera gatibu mantentzeko eman zuten aitzakia zen ezen egoera txarrean zegoela, nolaz bere bizitza uharrian, eguna joan eguna etorri janaria lortzeko borrokan, askosaz gogorragoa zela kaiola barruan zuena baino.\\

  \midrule
  \cellcolor{lightgray}{\textbf{Párrafo}} &
  \cellcolor{lightgray}{\textbf{Paragrafoa}}\\
  
  \midrule
  Según Rob Yarrall, gerente del acuario, Inky no era infeliz allí, ya que ``los pulpos son animales solitarios. Él era sólo un chico curioso y querría saber lo que estaba pasando en el exterior. Esa era, simplemente, su personalidad''.&
  Rob Yarrall-en arabera, akuarioko gerentea, Inky bertan ez zen zorionge sentitzen, nolaz ``Olagarroak animalia bakartiak dira. Bera, soilik, jakiminez beteriko izakitxoa zen eta kanpoaldean zer gertatzen zen jakin nahi zuen. Hori zen, sinpleki, bere pertsonalitatea''.\\

  \midrule
  Yarrall reconoce que el pulpo es un sujeto con personalidad, pero no es capaz de reconocer que, si de verdad su cautiverio le hacía tan feliz, no habría estrujado todo su cuerpo para salir del tanque a través de un pequeño agujero, no habría recorrido varios metros por el suelo, fuera del agua, ni se habría introducido por la tubería de 15 centímetros de diámetro que, ``casualmente'', dirigía directamente al océano.&
  Yarral-ek aintzatesten du ezen olagarroa pertsonalitatedun subjektua dela, baina ez da gai aintzatesteko ezen, baldin egiatan bere gatibutasunak hain zoriontsu egiten bazuen, ez zuela bere gorputz osoa horrela bihurrituko zulotxo txiki batetik barna tanketik irteteko, ez zituela hainbat metro lurrean arrastaka egingo, uretatik kanpo, eta ez zela ezta ere 15 zentimetroko diametroa zuen hodi batean sartuko, zeinak, ``kasualitatez'', zuzenean ozeanora bideratuko baitzuen.\\

  \midrule
  Con suerte, porque ahí se perdió el rastro de agua que encontraron las empleadas del acuario, Inky habría superado los 50 metros de longitud de esa tubería y habría logrado volver al mar.&
  Zortea lagun, zeren akuarioko langileek aurkitutako ur aztarna han galdu baitzuten, Inky-k 50 metro luzeko hodi hura gaindituko zuen eta itsasora itzultzea lortuko.\\

  \midrule
  Con suerte, pero no por casualidad.&
  Zortea lagun, baina ez kasualitatez.\\

  \midrule
  También cuesta creer que fuera casual que eligiera para escapar el momento en el que se estaban realizando labores de mantenimiento y se había dejado un hueco en la parte superior de su recinto, tal y como narra The Washington Post.&
  Baita ere zail da sinistea ezen kasualitatea izan zenik ihes egiteko aukeratutako unea noiz-eta bertan mantenu lanak gauzatzen ari zirenekoa izana, zeinarengatik itxituraren goikaldean zulo hori ireki baitzen, nola The Washington Post-ek narra hala ekarri delarik hona.\\

  \midrule
  A pesar del grado de planificación que sugiere la fuga de Inky, no vamos a pararnos a analizar, como han hecho los medios de masas, las extensas pruebas de la inteligencia y agilidad de los pulpos.&
  Inky-ren ihesaldiak plangintzatze maila handia iradokitzen badu ere, ez gara hau aztertzera geldituko, nola masentzako hedabideek egin duten gisara, olagarroen adimen eta bizkortasunaren proba luze zabalak aztertzera.\\

  \midrule
  Pensamos que, sencillamente, utilizó sus cualidades, puso lo mejor de sí y aprovechó el momento, como hacen constantemente animales de todas las especies, con mayor o menor inteligencia, guiados por el común anhelo de libertad.&
  Pentsatzen dugu ezen, sinpleki, bere nolakotasunak erabili zituela, bere onena jarri zuela eta unea aprobetxatu zuela, nola espezie guztietako animaliek konstanteki egiten duten bezala, adimen handiago edo txikiagoa izan, askatasun irrika komunak gidatuta.\\

  \midrule
  \cellcolor{lightgray}{\textbf{Párrafo}} &
  \cellcolor{lightgray}{\textbf{Paragrafoa}}\\

  \midrule
  Yarrall insinúa que el acuario está abierto a acoger otro pulpo ``si algún pescador se lo trae'', lo cual sugiere que toda la publicidad que están dando al caso les puede conseguir un nuevo esclavo de lo más barato, un negocio muy suculento para quienes utilizan a los demás animales como mercancía para ganar dinero.&
  Yarrall-ek insinuatzen du ezen akuarioa irekia dagoela beste olagarro bat abegi onez hartzera ``baldin arrantzale batek ekarriko balu'', zer honek iradokitzen baitu ezen kasuari ematen ari diren publizitate guztiak menpeko berri bat lor diezaiekeela erarik merkeenean, izan ere, negozio oso gustagarria baita gainontzeko animaliak dirua irabazteko merkantzia bailiran erabiltzen dituztenentzat.\\

  \midrule
  Pero Inky no era un objeto, sino un individuo que se hizo protagonista de su propia historia.&
  Baina Inky ez zen objektu bat, norbanako bat baizik bere historia propioaren protagonista bihurtu zena.\\

  \midrule
  Igual que lo es Blotchy, el otro pulpo aún atrapado en el mismo tanque, e igual que los más de cien animales cautivos en el Acuario de Napier, una cárcel de agua de un millón y medio de litros, donde también roban la vida a pirañas, tiburones, pingüinos, caimanes y peces tropicales, además de otros reptiles, aves y animales acuáticos.&
  Blotchy den bezalaxe, oraindik tanke berean harrapatua dagoen beste olagarroa, eta Napier-ko akuarioko gainerako ehundik gora animalia gatibuak bezalaxe, izan ere, tankeak, urezko kartzela bat denak eta milioi eta erdi litro dituenak, bizitza lapurtzen die piraña, marrazo, pinguino, kaiman eta arrai tropikalei ere, nola narrasti, hegazti eta uretako animaliei.\\

  \midrule
  \cellcolor{lightgray}{\textbf{Párrafo}} &
  \cellcolor{lightgray}{\textbf{Paragrafoa}}\\
  
  \midrule
  ``Nunca se sabe, siempre existe la posibilidad de que Inky pueda volver con nosotros'', declara el responsable del acuario, como si con él no fuera toda esta injusticia.&
  ``Ezin da inoiz jakin, beti existituko da posibletasuna ezen Inky guregana itzul dadin'', adierazi du akuarioko arduradunak, injustizia guzti hau berarekin joango ez balitz bezala.\\

  \midrule
  Sólo nos queda la esperanza de que, mientras él pronuncia esta terrible amenaza, Ynky esté flotando libre en el fondo del mar, y así le espere una larga vida sin jamás mirar atrás\ldots&
  Soilik itxaropena geratzen zaigu ezen, bera mehatxu beldurgarri hori ahoskatzen ari den bitartean, Inky itsasoaren hondoan aske flotatzen egongo dela, eta honela bizitza luze batek itxaron diezaiola nehoiz atzera begiratu beharrik izango ez duena\ldots\\

  \bottomrule
\end{longtable}
\end{center}


\begin{center}
  Artikulu hau ``Querer la libertad'' (((Askatasuna nahi izatea))) blogetik ekarria izan da, zeina Ihesaldiak atalean aurkitzen den. 2016/04/14. \\
  Bilatu $\rightarrow$ site:quererlalibertad.org ``inky, el pulpo que encontró''
\end{center}


\end{document}


