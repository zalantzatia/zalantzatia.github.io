\documentclass{article}
\usepackage[utf8]{inputenc}
\usepackage[T1]{fontenc}
\usepackage[basque]{babel}
\usepackage{hyperref}
\usepackage{graphicx}
\usepackage{longtable}
\usepackage{booktabs}
\usepackage[table]{xcolor}

\title{Mirtilaren ipuina}
\author{}
\date{}

\begin{document}

\maketitle

%\tableofcontents

\begin{center}
\begin{longtable}{|p{6cm}|p{6cm}|}
  \toprule
  \cellcolor{lightgray}{\textbf{Título}} &
  \cellcolor{lightgray}{\textbf{Izenburua}}\\
  
  \midrule
  El cuento de mirtila&
  Mirtilaren ipuina\\

  \midrule
  \cellcolor{lightgray}{\textbf{Párrafo}} &
  \cellcolor{lightgray}{\textbf{Paragrafoa}}\\
  
  \midrule
  Te voy a contar un cuento.&
  Ipuin bat kontatuko dizut.\\

  \midrule
  Una historia que será real, sólo si tú quieres que lo sea.&
  Istorio bat zeina benetakoa izango baita, soilik baldin, zuk hala izan dadin nahi baduzu.\\

  \midrule
  \cellcolor{lightgray}{\textbf{Párrafo}} &
  \cellcolor{lightgray}{\textbf{Paragrafoa}}\\
  
  \midrule
  No hace mucho tiempo, en una casa de campo, vivía una niña de seis años llamada Mirtila.&
  Duela ez asko, etxalde batean, Mirtila deitzen zen sei urteko neskato bat bizi zen.\\

  \midrule
  Sus padres estaban preocupados porque ahí no había más niños con los que jugar.&
  Bere gurasoak kezkaturik zeuden han ez zegoelako ume gehiagorik nortzuekin jolas zedin.\\

  \midrule
  Unas navidades sus padres le trajeron un regalo.&
  Eguberri batzuetan bere gurasoek opari bat ekarri zioten.\\

  \midrule
  Era una perrita, un cachorro de tres meses con un gran lazo azul atado a su cuello.&
  Txakurtxo eme bat zen, hiru hilabeteko txakurkume bat, lazo urdin handi bat lepotik loturik zuelarik.\\

  \midrule
  \cellcolor{lightgray}{\textbf{Párrafo}} &
  \cellcolor{lightgray}{\textbf{Paragrafoa}}\\
  
  \midrule
  Mirtila, inconscientemente, lo primero que hizo fue fijarse en sus ojos.&
  Mirtilak, inkontzienteki, lehenik egin zuen gauza, bere begietan arreta jartzea izan zen.\\

  \midrule
  Tenía una mirada que transmitía tristeza y temor a la vez, la misma mirada que tendría cualquier niño al que separan de su madre y lo llevan a un lugar desconocido.&
  Begirada bat zuen zeinak tristura eta ikara, biak batera, transmititzen baitzituen, edozein umek izango lukeen begirada berdina noiz-eta amarengandik banatzen dutenean eta leku ezezagun batera eramaten.\\

  \midrule
  \cellcolor{lightgray}{\textbf{Párrafo}} &
  \cellcolor{lightgray}{\textbf{Paragrafoa}}\\
  
  \midrule
  La niña, sin saber por qué, le quitó el lazo.&
  Neskatoak, zergaitikan ez zekielarik, lazoa kendu zion.\\

  \midrule
  La levantó con cuidado y la apoyó en su regazo, intentando que Jill, la perrita, notase el calor de su cuerpo.&
  Kontuz jaso eta bere magalean bermatu zuen, saiatzen zen ezen Jillek, txakurtxoak, bere gorputzaren beroa antzeman zezala.\\

  \midrule
  Mirtila nunca vio a Jill como un juguete, desde el primer momento la vio como una amiga.&
  Mirtilak ez zuen inoiz Jill jostailu bat bailitzan ikusi, lehen unetik lagun bat bailitzan ikusi zuen.\\

  \midrule
  Para ella Jill siempre sería alguien en quien confiar, un sentimiento recíproco que las unió hasta el final.&
  Beretzat Jill beti izango zen norbait zeinarengan fida bait-zitekeen, elkarrenganako sentimendu bat zuten azkeneraino batu zituena.\\  

  \midrule
  Desgraciadamente, cuando Mirtila entraba en la adolescencia, su amiga se fue para siempre dejando un vacío en su corazón que ningún humano pudo llenar jamás.&
  Tamalez, noiz-eta Mirtila nerabezarora heldu zenean, bere laguna betirako joan zen bere bihotzean hutsune bat utziz zeina inongo gizakik inoiz bete ahal ezin izango baitzuen.\\

  \midrule
  \cellcolor{lightgray}{\textbf{Párrafo}} &
  \cellcolor{lightgray}{\textbf{Paragrafoa}}\\
  
  \midrule
  El cariño que sentía hacia los animales la llevó a estudiar veterinaria.&
  Animalienganako sentitzen zuen laztan afektuak eraman zuen albaitaritza ikastera.\\

  \midrule
  Cuando ya habían pasado más de seis meses desde que entró en la facultad, un profesor les dijo que les llevaría a visitar el animalario.&
  Noiz-eta sei hilabete baino gehiago jada igaro zirenean noiztik-eta fakultatean sartu zenetik, irakasle batek esan zien ezen animaliategia bisitatzera eramango zituela.\\

  \midrule
  ``Por fin -pensó Mirtila- eso es lo que quería''.&
  ``Azkenean -pentsatu zuen Mirtilak- hau da nahi nuena''.\\

  \midrule
  Hasta el momento las clases le habían parecido demasiado frías, ya era hora de entrar en contacto con los animales que habían venido a ayudar.&
  Ordurarte klaseak hotzegiak iruditu zitzaizkion, jada bazen garaia animaliekin harremanetan jartzeko, zeintzuei laguntzeko baitzeuden han.\\

  \midrule
  \cellcolor{lightgray}{\textbf{Párrafo}} &
  \cellcolor{lightgray}{\textbf{Paragrafoa}}\\
  
  \midrule
  El profesor al día siguiente sorprendió a Mirtila cuando explicó que esos animales no estaban ahí porque estuviesen enfermos, sino que los tenían para realizar diferentes investigaciones.&
  Hurrengo egunean irakasleak Mirtila harritu zuen noiz-eta azaldu zuenean ezen animalia horiek ez zeudela berton gaixorik zeudelako, baizik eta ikerketa ezberdinak gauzatzeko zituztela gatibu.\\

  \midrule
  Tras acabar su introducción sobre lo que se realizaba ahí, entró en una nave y toda la clase le siguió.&
  Bere sar-hitzak amaitu ondoren zeintzuak bertan gauzatzen zenari buruzkoak baitziren, nabe batean sartu zen eta klase osoak atzetik jarraitu zion.\\  

  \midrule
  Entraron en una habitación oscura en la que sólo había una perra.&
  Gela ilun batean sartu ziren non soilikan txakur bat baitzegoen.\\

  \midrule
  Mirtila se quedó sin habla.&
  Mirtila mintzamenik gabe geratu zen.\\

  \midrule
  Toda la gente que tenía a su alrededor había desaparecido.&
  Inguruan zeukan jende guztia desagertu egin zen.\\

  \midrule
  No había tenido esa sensación, ni visto esa mirada desde hacía muchos años.&
  Ez zuen sentipen hori izan, ezta begirada hori ikusi ere duela urte askotatik.\\

  \midrule
  Desde que conoció a Jill, su mejor amiga.&
  Jill ezagutu zuenetik, bere lagunik onena.\\

  \midrule
  \cellcolor{lightgray}{\textbf{Párrafo}} &
  \cellcolor{lightgray}{\textbf{Paragrafoa}}\\
  
  \midrule
  Mientras sus compañeros de clase tomaban apresuradas anotaciones de las explicaciones del profesor por las distintas salas, ella intentaba comprender algo muchísimo más importante.&
  Noiz-eta gelakideek lasterka oharrak hartzen zituzten bitartean, zeintzuak irakaslearen azalpenarenak baitziren eta sala ezberdinetan zehar zenbiltzalarik hartuak, bera, gauza askozaz garrantzitsuago bat konprenitzen saiatzen zen.\\

  \midrule
  Aquella noche no pudo dormir, su cabeza seguía dando vueltas, seguía intentando explicarse por qué tenían a aquella perra aislada en ese cuarto y se preguntaba qué habrían hecho con ella para que tuviese aquella mirada.&
  Gau hartan ezin izan zuen lorik hartu, burua jira-biraka zebilkion, berekiko azalpen bila jarraitzen zuen, ea zergaitikan zeukaten txakur hura gela hartan isolaturik, eta bere buruari galdetzen zion harekin zer egin ote zuten ezen halako begirada izan zezan.\\

  \midrule
  \cellcolor{lightgray}{\textbf{Párrafo}} &
  \cellcolor{lightgray}{\textbf{Paragrafoa}}\\
  
  \midrule
  Durante varios días más estuvo pensando qué es lo que hacía que algunas personas tratasen así a los animales, pero nunca lo entendió.&
  Hainbat egun gehiagotan aritu zen pentsatzen zer zen eragiten zuena ezen pertsona batzuk animaliak horrela trata zitzaten, baina ez zuen inoiz ulertu.\\

  \midrule
  Después empezó a pensar en Jill.&
  Gero berarengan pentsatzen hasi zen, Jillengan.\\

  \midrule
  Recordó lo felices que habían sido las dos juntas y pensó lo mucho que la había echado de menos todo este tiempo.&
  Oroitu zuen zeinen zoriontsu izan ziren biak elkarrekin eta pentsatu zuen bere hutsunea zenbateraino sentitu zuen denbora guzti hartan.\\

  \midrule
  Por último se preguntó qué hubiese querido Jill que hiciera.&
  Azkenik, bere buruari galdetu zion zer zen Jillek nahiko zukeena ezen berak egin zezan.\\

  \midrule
  \cellcolor{lightgray}{\textbf{Párrafo}} &
  \cellcolor{lightgray}{\textbf{Paragrafoa}}\\
  
  \midrule
  Esa misma noche acabaron sus lamentos y las lágrimas de frustración dejaron de caer de sus ojos.&
  Gau hartan bertan amaitu ziren bere aieneak eta bere begietako frustraziozko malkoek erortzeari utzi zioten.\\

  \midrule
  Cuando ya no había nadie por la calle salió ella y con pasos decididos se dirigió hacia su facultad por última vez.&
  Noiz-eta kaletik jada inor ez zebilenean irten zen bera eta urrats erabakitsuak emanez bere fakultaterantz zuzendu zen azken aldiz.\\

  \midrule
  Pasó por delante del edificio en el que le habían dado clases y llegó hasta la nave.&
  Klaseak eman zizkioten eraikinaren aurretik igaro zen eta naberaino iritsi zen.\\

  \midrule
  Retiró con cuidado el cristal de una ventana y entró al pasillo.&
  Kontu handiz leiho bateko kristala erretiratu zuen eta korridorera sartu zen.\\

  \midrule
  \cellcolor{lightgray}{\textbf{Párrafo}} &
  \cellcolor{lightgray}{\textbf{Paragrafoa}}\\
  
  \midrule
  Sus pies la guiaron hasta la perra solitaria.&
  Bere oinek berarengana gidatu zuten, txakur bakartuarengana.\\

  \midrule
  La levantó en sus brazos, la abrazó y la apretó contra su pecho como años atrás había hecho con Jill.&
  Bere besoetan jaso, besarkatu eta bere bularren kontra estutu zuen urte batzuk lehenago Jillekin egin zuen bezala.\\

  \midrule
  Cinco minutos más tarde desaparecían juntas en la oscuridad de la noche.&
  Bost minutu beranduago, elkarrekin desagertu ziren gauaren iluntasunean.\\

  \midrule
  \cellcolor{lightgray}{\textbf{Párrafo}} &
  \cellcolor{lightgray}{\textbf{Paragrafoa}}\\
  
  \midrule
  Caminaban hacia una nueva vida, una nueva vida para las dos.&
  Bizitza berri baterantz zihoazen, bientzako bizitza berri bat izango zenerantz, ahizpa kideartekotasunean.\\
  
  \bottomrule
\end{longtable}
\end{center}


\begin{center}
  Itzulitako zatia dagoen liburuaren izenburua: \textbf{R-209}. 
  Lehen eginaldia (2009): Acción Vegana, Local Anarquista Magdalena y Sombras y Cizallas.
  Bigarren eginaldia (2014): ochodoscuatro ediciones. \\
  Bilatu $\rightarrow$ site:ochodoscuatroediciones.org R-209
\end{center}

\begin{center}
\begin{longtable}{|p{6cm}|p{6cm}|}
  \toprule
  \cellcolor{lightgray}{\textbf{R-209}} &
  \cellcolor{lightgray}{\textbf{R-209}}\\
  
  \midrule
  R-209 es el código utilizado por la policía sueca para informar de que se ha producido un delito por liberación animal&
  R-209, polizia suediarrak erabilitako kodea da infomatzeko ezen animalien askapenaren aldeko delitu bat jazo dela.\\
  
  \bottomrule
\end{longtable}
\end{center}



\end{document}


